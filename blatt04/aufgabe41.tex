\section*{Aufgabe 4.1}
\[h:coT_{DAut(X.2)}\rightarrow\mathcal{P}(X^*)\]
\[ h(t)=\mathcal{L}(coT_{DAut(X,2)},t)=\{w\in X^*|\beta^{coT}(\delta^{*coT}(t)(w))=1\}\]
h ist bijektiv genau dann, wenn h sowohl injektiv als auch surjektiv ist.\\
h ist injektiv bedeutet, dass jede Sprache maximal $DAut(X,2)$-Coterm hat.\\
Man kann mittels Struktureller Induktion über den Aufbau von t beweisen, dass zwei Terme, die die selbe Sprache beschreiben, identisch in ihrem Verhalten sind.\\
h ist surjektiv bedeutet. dass jede Sprache mindestens einen$DAut(X,2)$-Coterm hat.\\
Da ein $DAut(X,2)$-Coterm beliebig groß, sogar unendlich ist, lässt sich durch aufzählen mit $\delta$ und $\beta$ jede (auch unendliche) Menge an Worten Beschreiben.\\
\\
Da beides gegeben ist, ist h bijektiv.
